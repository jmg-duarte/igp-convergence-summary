\documentclass[a4paper, 11pt]{article}

\usepackage{cite}
\usepackage{fullpage}

\title{%
Achieving sub-second IGP convergence in large IP networks \\
	\large Article Summary}
\author{José Duarte}
\date{March 2019}

\begin{document}
\maketitle

\section{Introduction}
The paper describes and analyses the various factors influencing convergence time in IGP networks,
presenting a short description of the IS-IS protocol followed by a description of the convergence time.
Showing that the main problems achieving sub-second convergence lie on the Routing Information Base (RIB) and
Forwarding Information Base (FIB) update, 
being able to reduce its influence by introducing prefix reduction during the design stages, 
prefix prioritization during RIB/FIB updates and incremental updates to the FIB.

A simulation model was used to study the convergence time in larger networks concluding that the sub-second convergence goal is achievable on ISP scale networks.

\section{IS-IS Protocol}
In the IS-IS protocol the router exchanges HELLO PDUs with its neighbors to determine its local topology, 
afterwards it will flood a link-state packet (LSP) describing its local topology, 
this packet will contain at least the identifier of its neighbors.

For broadcast networks, IS-IS routers will elect a router to "represent" the broadcast network, 
the router will then generate a LSP describing this network and all attached routers.

Two situations may force the flooding of the LSP, 
the information contained in the LSP changed, meaning that a new LSP must be generated and flooded, 
or the LSP lifetime ended and it must be flooded again.
In order to ensure reliability when flooding the network, each LSP is acknowledged on each link.

When a router receives a LSP describing a topology change, it updates the Link State Database (LSDB), 
this event triggers the update of the RIB
which in turn triggers the update of the FIB.
In order to update the RIB a new Shortest Path Tree (SPT) 
must be computed based on the information contained in the LSDB.

\section{Convergence Time Components}
The convergence time can be characterized as $D + O + F + SPT + RIB + DD$ where:
$D$ is the time to detect a link failure, 
$O$ the time taken to originate a LSP describing the new topology,
$F$ the complete flooding time from the node detecting the failure to the rerouting nodes that require a FIB update to bring the network to a consistent forwarding state, 
$SPT$ the SPT computation time,
$RIB$ the time taken to update the RIB and the FIB and 
$DD$ which represents the time it takes to distribute the FIB updates to the linecards when using a distributed router architecture.

\subsection{Router Architecture, Processor Performance, Operating System}
One of the main bottlenecks on convergence time is the time taken updating the RIB and FIB components,
it is clear that the faster the processor, the faster the convergence.

A distributed router architecture with hardware packet processors is presented as a very well suited solution to
the problem given that the the CPU (RP) is able to dedicated all power to the control plane operation.
Handling all routing operations and delegating to the linecards the write of the FIB updates to the hardware packet processors.

The operating system (OS) running on the RP and the LineCard (LC) CPU's  
implements a process scheduler with multiple priorities and preemption capabilities, 
allowing for the IS-IS process to be scheduled immediately upon link failure.

During convergence on a distributed platform, two processes share the CPU: 
the IS-IS process to update the RIB and the FIB and the process distributing the FIB updates to the LC CPU's.

% Not sure about this paragraph
Given that the main bottleneck is present in the RIB update, 
the IS-IS process will start by updating the prefixes with the higher priority, 
in order to ensure the update of the LC's a quantum is used, 
when the quantum is over the IPC process is scheduled, distributing the updates to the LC's, going back to the prefix update when the quantum is over, repeating the process.

\subsection{Link Failure Detection}
The use of Packet over SDH/SONET (POS) links is pointed as being one of the major enablers of sub-second IGP convergence given their ability to detect failures in tens of milliseconds.

Mechanisms present in SDH and SONET allow the LC hardware to detect failure in less than 10 milliseconds.
When a failure is detected an high-priority interrupt is fired causing a POS routine to be executed,
which enforces a user-defined hold-time delaying the communication of the failure to the central CPU and allowing for SDH/SONET protection to occur.
If protection is not available, the failure is immediately signaled to the common CPU, updating the interface and scheduling IS-IS for reaction.

The measurements confirm that in most cases failure detection takes less than $20ms$, under heavy load, 
although rare, it is possible that another process was owning the CPU when the LC detected a failure,
in the worst-case, the process has the same priority as IS-IS, 
it was scheduled just before the failure and busy enough to consume the its full quantum.

When link or path alarms are cleared, 
timers are used to hold down the interface for an additional $10s$ before informing the routing protocols, 
ensuring robustness against unstable situations.

\subsection{LSP Origination}
In order to achieve a rapid convergence while avoiding the generation of unnecessary LSP's, generation timers with fixed values have been used,
these values have been set to limit the overhead during network instability, 
however they impact the convergence time by introducing unnecessary wait time when the network is stable.

Introducing dynamic timers to control LSP generation solves the problem.
Adapting to the current network conditions, 
with short generation times when the network is stable, allowing fast information exchange,
and increasing exponentially when the network is unstable,
introducing overhead to allow the network to settle down.

\subsection{Flooding}
The flooding time from the Failure node to the Rerouting nodes is the sum at each hop of the bufferization, serialization, propagation and the ISIS processing time.

Serialization and bufferization is considered to be negligible and only the propagation delay impact on convergence is measured in the simulation.

There are several optimizations that can be implemented for the fast-flooding process.

For a single-threaded IS-IS implementation the LSP must be flooded before the RIB is updated, given the time the latter can take jeopardizing the network convergence when the local node is not the only rerouting node.

The IS-IS specification suggests a $33ms$ pacing timer which is considered outdated, given the evolution in CPU and link speeds over 15 years, and damaging to the convergence time.
Fast flooding is introduced as a solution to the pacing problems, bypassing pacing on LSPs describing a new link-state change event and applying pacing on Refresh and TE LSPs. 
However, in the case of link flapping bursts of LSPs must be avoided, in order to do so,
burst size is controlled and pacing is re-applied when a configurable certain amount of LSPs has been flooded over a configurable certain amount of time.

The measurements for percentile-90, 95 and 100 were respectively $2ms$, $28ms$ and $52ms$.
In practice, the worst-case scenario should be neglected due to the low probability of occurrence,
given that for it to really occur, this worst-case must occur in the multiple parallel paths that exist in a meshed network.

It is concluded that the time to flood a single LSP is negligible compared to the sub-second convergence objective.

\subsection{SPT Computation}
The dynamic timers from the section 3.3 were also applied for the SPF calculations allowing for IGPs to be tuned such that when the network is stable, 
the timers will be short, reacting to topology changes within a few milliseconds, 
when the network is unstable SPF recalculation timers will increase in order to throttle the rate of response to network events.

This ensures fast convergence when the network is stable and only incurs moderate overhead when the network is unstable.

In IGP networks designed for fast convergence it is best practice to minimize the number of nodes in the topology.

An important optimization to SPF is the incremental SPF (iSPF).
iSPF analyses the impact of new LSPs on the previously computed SPF and minimizes the number of computations required updating only the necessary parts of the SPT.
For failures in smaller subsets of the network, the bigger the iSPF computation gain when compared to a full SPF.

Results show that the SPT computation can be executed in tens of milliseconds, without compromising network stability.

\subsection{RIB and FIB update}
The RIB/FIB update duration is linearly dependent on the number of modified prefixes.
As expected, the update time depends primarily on processor performance as this is the main bottleneck in the convergence process.
It is also shown that the time taken per RIB/FIB update is $\sim146\mu$s.

In order to minimize the RIB/FIB update component three approaches can be taken: 
implementing a network design rule to minimize the number of IGP prefixes, protocol and implementation optimization to allow the prioritization of some prefixes during the update and the optimization of the table management code.
The paper discusses the first two.

For a large ISP there are usually a few hundred important prefixes, the other only links interconnecting routers and since these prefixes are rarely used as destination addresses they mostly introduce overhead to the RIB/FIB updates.

Introduction prefix prioritization solves the problem, the important prefixes will be updated first making
the worst-case update duration scale based on the number of important prefixes instead of all prefixes.

\section{Simulation}
\subsection{Router model}
The link failure component was modeled in a way that both routers will not detect the failure at the same time.
For low delay links the lab measurements were used and for high delay links the time it takes to propagate the failure detection signal from location A to another random location. % ??

The LSP generation timers were not modeled and the simulator allowed the router to flood its LSP immediately.

The exponential backoff mechanism from section 3.5 was also modeled and it is configure with three parameters: $spf\_initial\_wait$, $spf\_exponential\_increment$ and $spf\_maximum\_wait$.

The SPT computation time was modeled as a function of the number of nodes in the network with some jitter in order to take into account other processes that may be running on the router CPU.
Only the full SPT computation was considered and not its incremental variants.

To model the time required to update the FIB, 
the number of prefixes whose FIB entries have changed and then multiplying the number of FIB updates with the time required to update one entry.
The simulator models two kinds of FIB updates: $static$ and $incremental$.

\subsection{Convergence time}
When there are no failures within the network, the routing is consistent, however on a link failure, the routers using the failed link need to update their FIB, while the FIBs are being updated the routing may not be consistent anymore.

The $instant\ of\ convergence$ is defined as the last moment at which the routing becomes and remains consistent.

\section{Simulation Results}
The tested networks were the GEANT network, a highly meshed network containing a lot of redundancy at its core, 
and the backbone nodes of a world wide Tier-1 ISP, containing about 200 routers.

For the GEANT network each POP is composed of a single router, for the Tier-1 ISP each POP is usually composed of two core routers, interconnected with redundant links, and several aggregation and access routers.

\subsection{IGP convergence after link failures}
For GEANT the failure of all links was simulated, 
while for the ISP the 50 links carrying the largest router to router paths were brought down.

When a link fails, the two router attached to it detect the failure and originate a new LSP. 
With the two-way connectivity check, 
a link is considered as having failed as soon as one of the two LSPs has been received by a rerouting router.

The results show that the sub-second convergence objective is easily met in the GEANT network.

For a larger network such as the Tier-1 SP, as expected, the convergence time is also larger when comparing to the GEANT network.
This difference can be explained by three factors.
The Tier-1 SP link delays are larger, contains more prefixes than GEANT and the larger number of nodes present in the Tier-1 SP leads to a longer SPT computation.

The impact of the link propagation delays is shown to be reduced due to the key factors of the convergence being SPF and FIB update times.

Improvements upon the convergence time were obtained when reducing the number of advertised prefixes by each router. When advertising one prefix per router, convergence time is halved for nearly all the considered failures in the Tier-1 SP.

\subsection{IGP convergence after router failures}
ISPs also need to face the correlated link and router failures, in order to model such failures it is considered all links attached to a router fail at the same time.
For GEANT the failure of all routers was simulated and for the Tier-1 SP network the failure of the 23 routers connected to the 50 most loaded links of the network was simulated.

For the failure of several links the first LSP received by a router is not always sufficient to describe the entire failure. In the case of a router failure the LSPs of all neighbors of the failed router might be needed to correctly update the FIB.

To evaluate convergence time in case of a router failure it is firstly considered a configuration corresponding to IS-IS routers that were not optimized for fast convergence.

The results show that for some router failures, the GEANT network can converge faster if the $spf\_initial\_wait$ parameter is set to $100ms$ than when set to $25ms$ given that for the latter some routers had to update their FIB twice in order to allow the routing to converge.

However, for the Tier-1 SP, the results show that the sub-second goal is only achieved for $60\%$ of the router failures and convergence time can reach $1400ms$.
The problem is that some routers will update their FIB before receiving the LSPs from all the failed router neighbors.

In order to solve this problem, it must be assured that the routers will only start the SPT computation after receiving all the LSPs describing the failure, this is made possible using the fast-flooding mechanism, being possible to achieve sub-second results just using static FIB updates and fast-flooding. 
Results also show that it is possible to improve on convergence time by replacing static FIB updates by incremental ones, reducing the time to update 10 prefixes on one router from $220ms$ (static updates) to $1ms$.

Simulation results from figure 9 (in the paper) show that combining a low $spf\_initial\_wait$ and a low $spf\_exponential\_increment$ yields the best IGP convergence times.

The simulations show that sub-second convergence can be met in large SP networks with a number of measures:
Not applying the pacing timer to urgent LSPs, flooding urgent LSPs before recomputing their SPT and updating the FIB, thus it is recommended to use fast-flooding for fast convergence.
The router should only modify the FIB entries affected by failures and prefix prioritization should be used to let the most important ones be updated first, an incremental algorithm should be used to update the SPT in order to reduce convergence time further and finally the configuration of the $spf\_initial\_wait$ on the network routers depends on expected failure types.

\section{Conclusion}
The paper presents an overview of the IS-IS protocol and points out why convergence times should be lowered. 
Building up on several improvements over the field and technology itself.
It presents the main factors of the convergence time and analyses each one and its impact presenting improvements if possible and solutions to problems that arise throughout, proving that sub-second convergence can be achieved for ISP-scale network without compromising on stability.

\end{document}