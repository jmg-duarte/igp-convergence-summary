\documentclass[a4paper, 11pt, UTF8]{article}

\usepackage{cite}
\usepackage{fullpage}

\title{%
Achieving sub-second IGP convergence in large IP networks \\
	\large Article Summary}
\author{José Duarte}
\date{March 2019}

\begin{document}
\maketitle

\section{Introduction}
In link-state protocols, each router maintains a database describing the Autonomous System's (AS) topology, which is referred to as the link-state database.
Each router that participates in the AS has an identical database.
Each piece in the database is a particular router's local state (e.g., the router's usable interfaces and reachable neighbors).
The router distributes its state through the AS by flooding.

Based on the link-state database, each router constructs a tree of shortest paths with itself as the root.
The resulting shortest-path tree gives the route to each destination in the AS.
Externally derived routing information will appear as leaves in the tree.

For several equal-cost routes to a certain destination, traffic will be equally distributed among them.
The cost of a route is described by a single dimensionless metric.\cite{J.Moy1998}

\subsection{Open Shortest Path First}
The Open Shortest Path First (OSPF) is a routing protocol for IP networks belonging to the group of Interior Gateway Protocols (IGPs), meaning that it will distribute routing information between routers belonging to a single AS.
It is based on link-state (LS) or Shorttest Path First (SPF) technology.

The OSPF protocol was designed expressly for the TCP/IP internet environment. 
It includes explicit support for Classless Inter-Domain Routing (CIDR) and tagging of externally-derived routing information.

All OSPF protocol exchanges are authenticated, meaning that only trusted routers can participated in the AS routing.

Allowing sets of networks to be grouped together into an area, the topology of such area is hidden from the rest of the AS.
This hiding mechanism also allows for a significant reduction in routing traffic.
The routing within such area will be determined by the area's own topology, lending the area protection from bad routing data.

OSPF uses the address from the IP packet header to route the packet through the network.
IP packets are routed "as is", meaning no changes are made as they transit the AS.

The OSPF is a dynamic routing protocol.
It is able to quickly detect topological changes in the AS and calculated loop-free routes after a convergence period.
This period is short and involves a minimum of routing traffic.\cite{J.Moy1998}

\subsection{IS-IS}
The Intermediate System to Intermediate System protocol partitions the network into "routing domains".
Boundaries for such routing domains are defined by network management, setting some links to be "exterior links".
Routing messages will not be sent on links marked as "exterior".

IS-IS routing makes use of two-level hierarchical routing. 
Each routing domain is partitioned into areas.
\begin{description}
  \item[$\bullet$ Level 1] - Routing within an area, level 1 routers will have a manually configured area address and refuse to become neighbors of other routers not belonging in the same area.
  \item[$\bullet$ Level 2] - Routing towards areas, not concerned with their internal structure, they may also be a level 1 router in one area.
\end{description}

Level 1 routers know the topology of their area, however they do not know the identity of the routers or destinations outside their area.
Level 1 routers must forward all their traffic for destinations outside of their area to a Level 2 router in their area.

Correspondingly, level 2 routers know the level 2 topology and reachable addresses via the level 2 router.
However, they do not need to know the topology of level 1 areas, except to the extent that a level 2 router may also be a level 1 router within a single area.
Only level 2 routers can exchange data packets or routing information directly with external routers located outside of the routing domains.\cite{Callon1990}

\

\bibliography{convergence}
\bibliographystyle{acm}

\end{document}